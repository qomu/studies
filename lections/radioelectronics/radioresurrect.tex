% !TEX encoding = UTF-8 Unicode
\documentclass[a4paper,12pt]{report}
\newcommand\hmmax{0}
\newcommand\bmmax{0}
		
%Math packages
\usepackage{amsmath, amsfonts, amssymb, amsthm, mathtools}
\usepackage{icomma}
\setcounter{MaxMatrixCols}{20}
%Math packages

%SetFonts
\usepackage{euscript}	 % Шрифт Евклид
\usepackage{mathrsfs} % Красивый матшрифт
\usepackage{dsfont} % Жирный шрифт для множеств чисел
%\usepackage{wasysym} % Для всяких символов
\usepackage{ stmaryrd } %Left \mapsto
\usepackage[T1, T2A]{fontenc}
\usepackage[utf8x]{inputenc}
\usepackage[english, russian]{babel}
%SetFonts

%Graphics
\usepackage{graphicx}
\usepackage{wrapfig}
%Graphics

%Tabular
\usepackage{array, tabularx, tabulary, booktabs}
\usepackage{longtable}
\usepackage{multirow}
%Tabular

%Additional
\newcommand*{\hm}[1]{#1\nobreak\discretionary{}
	{\hbox{$\matsurround=0pt #1$}}{}}
%Additional

%%% Страница
\usepackage{extsizes} % Возможность сделать 14-й шрифт
\usepackage{geometry} % Простой способ задавать поля
\geometry{top=20mm}
\geometry{bottom=20mm}
\geometry{left=15mm}
\geometry{right=15mm}
%
\usepackage{fancyhdr} % Колонтитулы
%\pagestyle{fancy} % Полный набор колонтитулов
\renewcommand{\headrulewidth}{0mm}  % Толщина линейки, отчеркивающей верхний колонтитул
%\lfoot{Нижний левый}
%\rfoot{Нижний правый}
%\rhead{Верхний правый}
%\chead{Верхний в центре}
%\lhead{Верхний левый}
% \cfoot{Нижний в центре} % По умолчанию здесь номер страницы

\usepackage{setspace} % Интерлиньяж
%\onehalfspacing % Интерлиньяж 1.5
%\doublespacing % Интерлиньяж 2
%\singlespacing % Интерлиньяж 1

\usepackage{lastpage} % Узнать, сколько всего страниц в документе.

\usepackage{soul} % Модификаторы начертания

\usepackage{hyperref}
\usepackage[usenames,dvipsnames,svgnames,table,rgb]{xcolor}
\hypersetup{				% Гиперссылки
    unicode=true,           % русские буквы в раздела PDF
    pdftitle={Заголовок},   % Заголовок
    pdfauthor={Автор},      % Автор
    pdfsubject={Тема},      % Тема
    pdfcreator={Создатель}, % Создатель
    pdfproducer={Производитель}, % Производитель
    pdfkeywords={keyword1} {key2} {key3}, % Ключевые слова
    colorlinks=true,       	% false: ссылки в рамках; true: цветные ссылки
    linkcolor=red,          % внутренние ссылки
    citecolor=green,        % на библиографию
    filecolor=magenta,      % на файлы
    urlcolor=cyan           % на URL
}

%\renewcommand{\familydefault}{\sfdefault} % Начертание шрифта



%%%Библиография
\usepackage{cite} % Работа с библиографией
%\usepackage[superscript]{cite} % Ссылки в верхних индексах
%\usepackage[nocompress]{cite} % 
\usepackage{csquotes} % Еще инструменты для ссылок


\usepackage{multicol} % Несколько колонок

\usepackage{tikz} % Работа с графикой
\usepackage{pgfplots}
\usepackage{pgfplotstable}
\usepackage{mathrsfs}
\usetikzlibrary{arrows}

\usepackage{circuitikz}


%reset equation counter after section
\usepackage{chngcntr}

\counterwithin*{equation}{section}
\renewcommand\theequation{\arabic{equation}}
\renewcommand\thesection{\arabic{section}}
%


%Code listings support
\usepackage{listings}
\lstloadlanguages{[5.2]Mathematica}

\author{Сочава Александр Андреевич}
\title{Радиотехнические цепи и сигналы}
\date{2018 год, весенний семестр}
%\date{}							% Activate to display a given date or no date

\begin{document}

\makeatletter
\@addtoreset{chapter}{part}
\makeatother  

\maketitle
\newpage
%\section{}
%\subsection{}
\setcounter{section}{0}

\section{Литература}
\textbf{Основная:}\\
Атабеков: Теоретические основы электроники. Линейные электрические цепи, главы "Четырёхполюсники" и далее.\\
Зернов, Карпов: Теория радиотехнических цепей.\\
\textbf{Дополнительная:}\\
Манаев: Основы радиоэлектроники (устарела).\\
Титце, Шек: Полупроводниковая схемотехника.\\
Алексеенко: Применение прецизионных полупроводниковых микросхем.\\

\section{Четырёхполюсники}
\subsection{Основные определения. Классификация четырёхполюсников.}
Четырёхполюсник - некоторое устройство, которое имеет 4 вывода и внутренняя структура которого не важна. Как правило, входные выводы изображаются слева, а выходные - справа.

\begin{center}
\begin{circuitikz} \draw
(0,0) node[draw,minimum width=2cm,minimum height=2.4cm] (c) {}
($(c.west)!0.5!(c.north west)$) coordinate (cnw)
($(c.west)!0.5!(c.south west)$) coordinate (csw)
($(c.east)!0.5!(c.north east)$) coordinate (cne)
($(c.east)!0.5!(c.south east)$) coordinate (cse)
(cnw) to[short,-o] ++(-1,0)
(cne) to[short,-o] ++(1,0)
(cse) to[short,-o] ++(1,0)
(csw) to[short,-o] ++(-1,0)
(cnw) node[above left] {$1$}
(csw) node[above left] {$1'$}
(cne) node[above right] {$2$}
(cse) node[above right] {$2'$}
;
\end{circuitikz}
\end{center}

Все устройства, как правило, имеют одну общую точку. Часто заземляют один входной и один выходной выводы. Таким образом, они будут иметь одинаковый нулевой потенциал.\\
(картинка)\\
Это регулярный четырёхполюсник, в такой схеме 3 независимых вывода.\\
Четырёхполюсники бывают линейные и нелинейные (по признаку входящих в него элементов).\\
Эквивалентные схемы четырёхполюсников:\\
(картинка)\\
\textbf{Активный четырёхполюсник} - четырёхполюсник, содержащий в себе зависимые или независимые источники энергии. Эти источники могут компенсировать друг друга так, чтобы напряжение на незапитанном четырёхполюснике было равно 0, тогда четырёхполюсник называется \textbf{неавтономным}, в противном случае - \textbf{автономным}.\\
\textbf{(Не)зависимый источник} энергии - источник, параметры которого (не) зависят от внешнего питания.\\
\textbf{Пассивный четырёхполюсник} - четырёхполюсник, не содержащий в себе источники электрической энергии, или эти источники взаимно компенсируются для любого внешнего напряжения (иначе говоря, четырёхполюсник эквивалентен пассивному без источников энергии).\\
\textbf{Эквивалентность} нескольких \textbf{четырёхполюсников} - возможность взаимной замены их в электрической цепи без изменений токов и напряжений в остальной её части.\\
\textbf{Симметричный четырёхполюсник} - четырёхполюсник, перемена местами входных и выходных выводом в котором не меняет токов и напряжений в его цепи. В противном случае четырёхполюсник называется \textbf{несимметричным}.\\
\textbf{Обратимый четырёхполюсник} - четырёхполюсник, для которого напряжение на входе к току на выходе (=передаточное сопротивление) не зависит от того, какая пара выводов является входом, а какая - выходом (то есть, для него выполняется теорема обратимости). В противном случае четырёхполюсник называется \textbf{необратимым}.\\
Из симметричности четырёхполюсника следует его обратимость, обратное неверно.\\
\textbf{Смысл теории} четырёхполюсников - найти токи и напряжения на входе и выходе четырёхполюсника по его обобщённым параметрам.\\
\subsection{Система уравнений четырёхполюсника}
Обозначим токи и напряжения на входе и выходе четырёхполюсника следующим образом:\\
(картинка)\\
\\
Вариант с токами $I_1$ и $I_2$ называется прямой передачей (для формы $||\textbf{A}||$), с токами $I_1'$ и $I_2'$ - обратной передачей (для формы $||B||$). Также часто используется третий вариант - токи $I_1$ и $I_2'$ (для форм $||H||$, $||G||$, $||\textbf{Y}||$, $||Z||$).\\
Соотношение между напряжениями и токами на входе и выходе может быть записано в виде одной из следующих систем уравнений:
\begin{enumerate}
  \item Форма $||Y||$ (токи втекают, выражаем токи через напряжения), коэффициенты - проводимости:
  \begin{equation}
  \left\{
\begin{aligned}
&I_1=Y_{11}U_1+Y_{12}U_2\\
&I_2'=Y_{21}U_1+Y_{22}U_2 \notag
\end{aligned}
\right.
\end{equation}

  \item Форма $||Z||$ (токи втекают, выражаем напряжения через токи), коэффициенты - сопротивления:
  \begin{equation}
    \left\{
\begin{aligned}
&U_1=Z_{11}I_1+Z_{12}I_2'\\
&U_2=Z_{21}I_1+Z_{22}I_2'  \notag
\end{aligned}
\right.
\end{equation}
Для обратимого четырёхполюсника $Z_{21}=Z_{12}$.\\
Для симметричного четырёхполюсника $Z_{11}=Z_{22}$.\\

  \item Форма $||H||$ (токи втекают, выражаем входное напряжение и выходной ток):
  \begin{equation}
  \left\{
\begin{aligned}
&U_1=H_{11}I_1+H_{12}U_2\\
&I_2'=H_{21}I_1+H_{22}U_2  \notag
\end{aligned}
\right.
\end{equation}
  \item Форма $||G||$ (токи втекают, выражаем входной ток и выходное напряжение):
  \begin{equation}
  \left\{
\begin{aligned}
&I_1=G_{11}U_1+Y_{12}I_2'\\
&U_2=G_{21}U_1+Y_{22}I_2' \notag
\end{aligned}
\right.
\end{equation}
\\
  \item Форма $||A||$ (зависимость входных параметров от выходных, прямая передача):
  \begin{equation}
\left\{
\begin{aligned}
&I_1=A_{11}U_2+A_{12}I_2\\
&I_2'=A_{21}U_2+A_{22}I_2 \notag
\end{aligned}
\right.
\end{equation}
  \item Форма $||B||$ (зависимость выходных параметров от входных, обратная передача): 
  \begin{equation}
  \left\{
\begin{aligned}
&U_2=B_{11}U_1+B_{12}I_1'\\
&I_2'=B_{21}U_1+B_{22}I_1'  \notag
\end{aligned}
\right.
\end{equation}
\end{enumerate}

\textbf{Подключение четырёхполюсников.}

(вставить картинки)
\begin{enumerate}
\item Каскадное: $\|A\|=\|A_1\|*\|A_2\|$


\item Последовательное: $\|Z\|=\|Z_1\|+\|Z_2\|$
\item Параллельное: $\|Y\|=\|Y_1\|+\|Y_2\|$
\item Последовательно-параллельное: $\|H\|=\|H_1\|+\|H_2\|$
\end{enumerate}

Согласованное подключение: 

Основные параметры: $K_i, K_u, K_p$ - коэффициенты усиления (передачи).
Для системы $y$-параметров ($Y_n=\frac{1}{Z_n}$):

  \begin{equation}
  \left\{
\begin{aligned}
&I_1=Y_{11}U_1+Y_{12}U_2\\
&I_2'=Y_{21}U_1+Y_{22}U_2 \\
& Y_\text{вх}=\frac{I_1}{U_1}=Y_11+Y_12\frac{U_2}{U_1}=\frac{\Delta_Y+Y_11Y_H}{Y_22+Y_H}
\end{aligned}
\right.
\end{equation}

Получим $Y_\text{вх}, Y_\text{вых}$:
  \begin{equation}
\begin{aligned}
& Y_\text{вх}=\frac{I_1}{U_1}=Y_{11}+Y_{12}\frac{U_2}{U_1}=\frac{\Delta_Y+Y_{11}Y_H}{Y_{22}+Y_H}\\
& Y_\text{вых}=\frac{I_2}{U_2}=Y_{22}-Y_{12}\frac{U_2}{U_1}=\frac{\Delta_Y+Y_{11}Y_H}{Y_{22}+Y_H}
\end{aligned}
\end{equation}


\subsection{Свойства трёхполюсников.}

Для четырёхполюсников выводы $1'$, $2'$ считаются обычно особыми - разность потенциалов между ними не влияет на параметры передачи. Тогда эту разность потенциалов можно принять равной $0$. (картинка)

\textbf{Матрица неопределённых проводимостей}
Считаем все токи втекающими, а четырёхполюсник - линейным. Тогда токи зависят от потенциалов линейно - выразим зависимость системой линейных уравнений.\\ 
  \begin{equation}
  \left\{
\begin{aligned}
I_1=Y_{11}U_1+Y_{12}U_2+Y_{13}U_3&\\
I_1=Y_{21}U_1+Y_{22}U_2+Y_{23}U_3&\\
I_1=Y_{31}U_1+Y_{32}U_2+Y_{33}U_3&
\end{aligned}
\right.
\end{equation}

  \begin{equation}
  Y=\left(
\begin{aligned}
& Y_{11} & Y_{21} & Y_{31}&\\
& Y_{12} & Y_{22} & Y_{32}&\\
& Y_{13} & Y_{23} & Y_{33}&
\end{aligned}
\right)
\end{equation}
Неизвестно 9 параметров. Предположим, что известны $Y_{11}-Y{22}$, найдём оставшиеся. Т.к. $I_1+I_2+I_3=0$, то при любых потенциалах:
  \begin{equation}
  \left\{
\begin{aligned}
0=Y_{11}+Y_{21}+Y_{31}&\\
0=Y_{12}+Y_{22}+Y_{32}&\\
0=Y_{13}+Y_{23}+Y_{33}&
\end{aligned}
\right.
\end{equation}

Скажем, что потенциал точки отсчёта - $\delta$. Тогда все напряжения будут иметь вид $U_i+\delta$. Для каждого $I_i$ придётся заключить, в этом случае, что сумма коэффициентов перед $\delta$ будет равна $0$.
  \begin{equation}
  \left\{
\begin{aligned}
0=Y_{11}+Y_{12}+Y_{13}&\\
0=Y_{21}+Y_{22}+Y_{323}&\\
0=Y_{31}+Y_{32}+Y_{33}&
\end{aligned}
\right.
\end{equation}

Получили $6$ уравнений, из них $5$ независимых. Значит, в матрице $Y$ можно будет определить все коэффициенты.\\
Если $U_3=0$, то останутся независимые уравнения для $I_1, I_2, I_3=-(I_1+I_2)$. Зачеркнём третьи столбец и строку.\\
Для $U_2=0$: вход - $3$, выход - $1$. Тогда:
\begin{equation}
  Y'=\left(
\begin{aligned}
& Y_{22} & Y_{32}&\\
& Y_{23} & Y_{33}&
\end{aligned}
\right)
\end{equation}
Перевернём матрицу, т. к. выход и и вход теперь изменены местами.
\begin{equation}
 Y =\left(
\begin{aligned}
&Y_{33} & Y_{23}&\\
&Y_{32}& Y_{22}&
\end{aligned}
\right)
\end{equation}

С помощью матрицы неопределённых проводимостей можно рассчитать Y-параметры в схемах ОК и ОБ для биполярного транзистора (в режиме малого сигнала - поэтому транзистор можно считать линейным четырёхполюсником). Аналогично для полевых транзисторов и ламп.

\subsection{Принцип усиления.}


\textbf{Усилителем} называют устройство, которое служит для преобразования \textit{информационных} сигналов и для которого выполнено условие: мощность сигнала на выходе больше, чем на входе.

\textbf{Усиление} можно определить как непрерывный процесс управления большим количеством энергии с помощью малых затрат энергии. Форма управляющей и управляемой энергий может быть различной: свотовой, механической, электрической. Будем говорить про усиление и усилители электрических сигналов. 

\textbf{Источник сигнала} - управлящий источник энергии.

\textbf{Входная цепь (вход)} - цепь, по которой поступает энергия. Так как мощность выходного сигнала должна быть больше мощности входного, источник дополнительной энергии обязателен по ЗСЭ.

\textbf{Выходная цепь (выход), нагрузка} - потребитель усиленных сигналов.

\textbf{Источник питания усилителя} или \textbf{основной источник питания} - источник управляемой энергии, преобразуемый усилителем в энергию усиленных сигналов. Вспомогательные источники могут запитывать усилительные элементы, например.

\subsection{Классификация усилителей.}
\begin{itemize}
\item По характеру усиливаемых сигналов:

а) Усилители гармонических сигналов  - для непрерывных квазипериодических сигналов, гармоническая составляющая которых меняется много медленнее нестационарных (переходных) процессов в усилителе.

б) Усилители импульсных сигналов. Нужно максимально сократить длительность нестационарных процессов.

\item По ширине полос частот и абсолютным значениям усиливаемых усилителем частот:

а)Усилители постоянного (очень низкочастотного) тока

б)Усилители переменного тока

в)Усилители высоких частот

г)..;

д) Усилители низких частот:

\item По отношению к полосе усиливаемых частот:

а)Широкополосные

б)Избирательные и селективные

в)Резонансные, с резонансной характеристикой

Если частота на входе и выходе одинаковые - прямое усиление. Если разные - усиление с 

изменением частоты.

\item По типу усиляющего элемента:

а) Полупроводниковые.

б) Ламповые.
\end{itemize}

\subsection{Нелинейные цепи в режиме постоянного тока}
К нелинейным элементам относятся диод (в том числе варикап), трансформатор (катушка с ферромагнитным сердечником).

Рассчитываем рабочую точку либо аналитически (по аппроксимации характеристик нелинейных элементов), либо экспериментально (сняв характеристики опытным путём):
\begin{itemize}
\item Графо-аналитический

Смотрим на пересечение графиков функций, полученных аналитическим приближением, или снятых ВАХ, обычно прямой и нелинейной кривой. Находим точки пересечения.

\item Графический (современный численный метод)

Можем складывать экспериментально построенные нелинейные графики, смотреть их пересечение.
\end{itemize}

Для сложных двуполюсников (тиристоров, туннельных диодов) решение может получиться неоднозначным (несколько точек пересечения) и неустойчивым (обычно устойчивыми получаются точки на возрастающих кусках ВАХ, с положительным дифференциальным сопротивлением). 

\textbf{Рабочая точка для биполярного транзистора}

Зависимости для токов транзистора
\begin{equation}
\begin{aligned}
& i_1=f(U_1, U_2)\\
& i_2=f(U_1, U_2)
\end{aligned}
\end{equation}
представляют собой поверхность. Можем порезать эту поверхность на сечения плоскостями $U_i=const$, получим кривые входной и выходной ВАХ.

Для входной ВАХ можем также построить нагрузочную прямую: $i(U_\text{вх})=E_\text{б}-R_\text{г}I_1$. Точки определяются параметрами $i_1, U_1, U_2$. Вообще графики семейства здесь не сильно отстоят друг от друга, и их можно заменить на один график.

Для выходной - то же самое, только: $i(U_\text{вых})=E_\text{э}-R_\text{н}. Точки определяются параметрами I_1$ $i_1, i_2, U_2$

Для определения рабочей точки нужно найти пересечение кривых: $i_1, U_2$

\textbf{Рабочая точка для полевого транзистора}

Для $U_1<0$ при $E_1=0$ входное сопротивление очень велико, поэтому $U_ \approx E_2$ ИМеет смысл рассматривать только выходные сигналы.

\subsection{Работа в режиме малого сигнала}
\textbf{Малый сигнал}: величина амплитуды переменного тока (напряжения) пренебрежимо мала по сравнению с постоянными токами и напряжениями в рабочей точке.
Если переменные сигналы по своему размаху много меньше токов и напряжений в рабочей точке, то такой сигнал называется малым. Недостаток определения - не работает в 0.

Данную схему можно разбить на цепь постоянного и переменного тока:
\begin{center}
\begin{circuitikz} 
\draw (0,1)  to[battery1] (0, 0);
\draw (0,1) to[american current source] (0, 2)
to[european resistor] (2, 2) to[D*] (2, 0)--(0,0);

\draw (4,2)--(4,1)  to[battery1] (4, 0);
\draw (4, 2) to[european resistor] (6, 2) to[D*] (6, 0)--(4,0);

\draw (8, 0)--(8,1) to[american current source] (8, 2)
to[european resistor] (10, 2) to[D*] (10, 0)--(8,0);

\end{circuitikz}
\end{center}

Решение для постоянной цепи было получено. Для переменной: можем разложить нелинейную ВАХ в рабочей точке (точке равновесия) в ряд Тейлора: $i=f(U_0)+f'|_{U_0} \Delta U +...\Delta i| \approx g\Delta U= \Delta U/R_\text{диф}$

То есть, мы рассматриваем эквивалентное сопротивление $R_\text{диф}=\dfrac{dU}{di}$ вместо диода:

\begin{center}
\begin{circuitikz} 
\draw (8, 0)--(8,1) to[american current source] (8, 2)
to[european resistor] (10, 2) to[european resistor, l=$R_\text{диф}$] (10, 0)--(8,0);
\end{circuitikz}
\end{center}

То же самое для трёхполюсников:
\begin{equation}
\begin{aligned}
& i_1=f(U_1, U_2)\\
& i_2=f(U_1, U_2)
\end{aligned}
\end{equation}
Рабочая точка найдена, придадим $U_1, U_2$ малые приращения:
\begin{equation}
\begin{aligned}
& i_1=f(U_{10}, U_{20})+\dfrac{\partial f_1}{\partial U_1}\Delta U_1+\dfrac{\partial f_1}{\partial U_2}\Delta U_2+...\\
& i_2=f(U_{10}, U_{20})+\dfrac{\partial f_2}{\partial U_1}\Delta U_1+\dfrac{\partial f_2}{\partial U_2}\Delta U_2+...
\end{aligned}
\end{equation}

\begin{equation}
\left\{
\begin{aligned}
& i_1=g_{11}\Delta U_1+g_{12}\Delta U_2\\
& i_2=g_{21}\Delta U_1+g_{22}\Delta U_2
\end{aligned}
\right.
\end{equation}
где $g_{11}=\dfrac{di_1}{dU_1}|_{U_{10}, U_{20}}$ и так далее - вещественные дифференциальные (малосигнальные) параметры. Параметры вещественные, потому что сигнал в транзисторе не запаздывает, в отличие от реактивных элементов. Но при быстрых изменениях нужно учитывать инерционность транзистора: $g_{ik} \Rightarrow Y_{ik}=g_{ik}+jb_{ik}$. Посчитаем $b_{ik}$ позже.

По-другому записав уравнения четырехполюсника, получим другие параметры (например, H-параметры). Их можно выражать друг через друга.

  \begin{equation}
  \left\{
\begin{aligned}
& i_1=Y_{11}U_1+Y_{12}U_2\\
& i_1-Y_{12}U_2=Y_{11}U_1\\
& U_1=\dfrac{1}{Y_{11}}i_1-\dfrac{Y_{12}}{Y_{11}}\\
& h_{11}=\dfrac{1}{Y_{11}}, h_{12}=-\dfrac{Y_{12}}{Y_{11}}
\end{aligned}
\right.
\end{equation}

\subsection{Малосигнальные параметры полевых и биполярных транзисторов}
\textbf{Полевой транзистор в схеме с общим истоком}

Согласно схеме $U_\text{ЗИ}<0, g_{11}=g_{12}, i_2=0$

$g_{21}=S \approx (2-5)10^{-3}$ См

$g_{22}=g_i \approx 10^{-5}$ См, $R_i \approx 10^5$ Ом

Проходные характеристики: $i_2(U_1) \text {при} U_2=const$. Их производная - крутизна.

\textbf{Биполярный транзистор в схеме с общим эмиттером}

В отличие от предыдущего случая присутствует входной ток $i_\text{б}$, то есть $g_11, g_12 \not =0$, но при напряжении питания $\approx 5-10$ В характеристики сливаются, и проводимость $g_{12}<0$ очень мала по сравнению с другими $g$. Диапазон значения около рабочей точки порядка одного вольта. По входным и выходным ВАХ находим $h$-параметры, из них по формулам перехода находим $Y$-параметры.

Таблицы $Y$ и $h$-параметров
\begin{tabular}{c|c|c}
  Тип& БТ & ПТ  \\ \hline
 $g_11$ & 0 & $10^{-3}$ \\
  $g_12$ & 0 & $10^{-6} \approx 0$\\
   $g_21$ & $(2-8)*10^{-3}$ & $(50-200)10^{-3}$ См\\
    $g_11$ & $10^{-2}$ & $10^{-3}$\\
\end{tabular}

\section{Классические схемы усилителей}
\subsection{Усилитель с общим эмиттером (НЧ без инерционности)}

\begin{center}
\begin{circuitikz} 
\draw (0,1)  to[battery1] (0, 0);
\draw (0,1) to[american current source] (0, 2)
to[european resistor] (2, 2) to[D*] (2, 0)--(0,0);
\end{circuitikz}
\end{center}

На НЧ параметры вещественны, режим установившийся считаем малосигнальным

\begin{equation}
Y=\left(
\begin{aligned}
& i_1=g_{11}\Delta U_1+g_{12}\Delta U_2\\
& i_2=g_{21}\Delta U_1+g_{22}\Delta U_2
\end{aligned}
\right)
\end{equation}
 --- не пишем $q_1$, надо $g_1$

Определение $Y (g)$ аналитическим способом:
Тепловой потенциал $V=\dfrac{kT}{q}$
\end{document}

