% !TEX encoding = UTF-8 Unicode
\documentclass[a4paper,12pt]{report}
\newcommand\hmmax{0}
\newcommand\bmmax{0}
		
%Math packages
\usepackage{amsmath, amsfonts, amssymb, amsthm, mathtools}
\usepackage{icomma}
\setcounter{MaxMatrixCols}{20}
%Math packages

%SetFonts
\usepackage{euscript}	 % Шрифт Евклид
\usepackage{mathrsfs} % Красивый матшрифт
\usepackage{dsfont} % Жирный шрифт для множеств чисел
%\usepackage{wasysym} % Для всяких символов
\usepackage{ stmaryrd } %Left \mapsto
\usepackage[T1, T2A]{fontenc}
\usepackage[utf8x]{inputenc}
\usepackage[english, russian]{babel}
%SetFonts

%Graphics
\usepackage{graphicx}
\usepackage{wrapfig}
%Graphics

%Tabular
\usepackage{array, tabularx, tabulary, booktabs}
\usepackage{longtable}
\usepackage{multirow}
%Tabular

%Additional
\newcommand*{\hm}[1]{#1\nobreak\discretionary{}
	{\hbox{$\matsurround=0pt #1$}}{}}
%Additional

%%% Страница
\usepackage{extsizes} % Возможность сделать 14-й шрифт
\usepackage{geometry} % Простой способ задавать поля
\geometry{top=20mm}
\geometry{bottom=20mm}
\geometry{left=15mm}
\geometry{right=15mm}
%
\usepackage{fancyhdr} % Колонтитулы
%\pagestyle{fancy} % Полный набор колонтитулов
\renewcommand{\headrulewidth}{0mm}  % Толщина линейки, отчеркивающей верхний колонтитул
%\lfoot{Нижний левый}
%\rfoot{Нижний правый}
%\rhead{Верхний правый}
%\chead{Верхний в центре}
%\lhead{Верхний левый}
% \cfoot{Нижний в центре} % По умолчанию здесь номер страницы

\usepackage{setspace} % Интерлиньяж
%\onehalfspacing % Интерлиньяж 1.5
%\doublespacing % Интерлиньяж 2
%\singlespacing % Интерлиньяж 1

\usepackage{lastpage} % Узнать, сколько всего страниц в документе.

\usepackage{soul} % Модификаторы начертания

\usepackage{hyperref}
\usepackage[usenames,dvipsnames,svgnames,table,rgb]{xcolor}
\hypersetup{				% Гиперссылки
    unicode=true,           % русские буквы в раздела PDF
    pdftitle={Заголовок},   % Заголовок
    pdfauthor={Автор},      % Автор
    pdfsubject={Тема},      % Тема
    pdfcreator={Создатель}, % Создатель
    pdfproducer={Производитель}, % Производитель
    pdfkeywords={keyword1} {key2} {key3}, % Ключевые слова
    colorlinks=true,       	% false: ссылки в рамках; true: цветные ссылки
    linkcolor=red,          % внутренние ссылки
    citecolor=green,        % на библиографию
    filecolor=magenta,      % на файлы
    urlcolor=cyan           % на URL
}

%\renewcommand{\familydefault}{\sfdefault} % Начертание шрифта



%%%Библиография
\usepackage{cite} % Работа с библиографией
%\usepackage[superscript]{cite} % Ссылки в верхних индексах
%\usepackage[nocompress]{cite} % 
\usepackage{csquotes} % Еще инструменты для ссылок


\usepackage{multicol} % Несколько колонок

\usepackage{tikz} % Работа с графикой
\usepackage{pgfplots}
\usepackage{pgfplotstable}
\usepackage{mathrsfs}
\usetikzlibrary{arrows}

\usepackage{circuitikz}


%reset equation counter after section
\usepackage{chngcntr}

\counterwithin*{equation}{section}
\renewcommand\theequation{\arabic{equation}}
\renewcommand\thesection{\arabic{section}.}
%


%Code listings support
\usepackage{listings}
\lstloadlanguages{[5.2]Mathematica}

\author{Кудряшова Татьяна Юрьевна (а. 247, 2 уч. корп)}
\title{Численные методы}
\date{2018 год, весенний семестр}
%\date{}							% Activate to display a given date or no date

\begin{document}

\makeatletter
\@addtoreset{chapter}{part}
\makeatother  

\maketitle


\newpage
%\section{}
%\subsection{}
\setcounter{section}{0}

\section{Математические вычисления}

1. Аналитические преобразования (символьные вычисления)
2. Численные расчеты - численные методы решения математических задач в численном виде.
\begin{itemize}
\item Решение систем линейных уравнений
\item Решение систем нелинейных уравнений (включает, например, $3x-e^x=0$)
\item Интерполирование и приближение функций
\item Инегрирование функций
\item Решение дифференциальных уравнений
\item Решение задач оптимизации
\item Решение уравнений в частных производных
\end{itemize}

\section{Вычислительный эксперимент}
\begin{itemize}
\item Выявление основных законов, влияющих на объект
\item Запись этих законов в виде уравнений
\item Численное решение задачи (алгоритм)
\item Программирование
\item Проверка, анализ, уточнение
\end{itemize}

\begin{center}
\textbf{Причины возможных ошибок}
\end{center}

\begin{itemize}
\item Неполнота знаний об объекте 
\item  Упрощение свойств исходного объекта
\item Погрешность метода (алгоритса)
\end{itemize}


\section{Способы представления чисел в памяти вычислительных устройств}
\begin{itemize}
\item Числа с фиксированной запятой (точкой) (стандарт $ISO/IEC\;TR\;18037$)
\item Числа с плавающей запятой (точкой) (стандарт $IEEE\;754$)
\end{itemize}

\textbf{Числа с фиксированной запятой}

Используются, когда нужна минимальная поддержка дробных чисел на целочисленном процессоре.
Плюсы:
\begin{itemize}
\item ускорение вычислений, когда не нужна большая точность
\item устойчивость к ошибкам округления
\item меньшая стоимость
\end{itemize}
Минусы:
\begin{itemize}
\item малый диапазон вещественных чисел
\end{itemize}

10. $1*10^2+0*10^1+3*10^0+6*10^{-1}+7*10^{-2}$\\
2. $101,011=1*2^2+0*2^1+1*2^0+0*2^{-1}+1*2^{-2}+1*2^{-3}$\\
r. $a=+-a_n r^na_{n-1} r^{n-1}+...a_0r^0+a_{-1}r^{-1}...$, $a_k$ - разряд, $r$ - основание системы

$x$ - число ($103,67$)\\
$x'$ - целочисленное представление ($10367$)\\
$z$ - вес младшего разряда ($-2$). Обычно фиксируется и не хранится.\\

\textbf{Двоично-десятичный код}
Каждый десятичный разряд записывается в виде 4-битного двоичного представления.
Пример: $310_{10}=0011\;0001\;0000$

Есть запрещенные состояния, которые невозможно перевести обратно в 10-систему.

\textbf{Числа с плавающей запятой}

Плюсы:

\begin{itemize}
\item можно использоватьбольший диапалон значений при той же относительной точности
\item меньшая погрешность численного метода
\end{itemize}

Минусы:

\begin{itemize}
\item возможны большие ошибки округления
\item вычислительные устройства существенно дороже
\end{itemize}

Мантисса (M) - целое число фиксированной длины, оперделяющее старшие разряды действительного числа. Всегда начинается с 1 (в двоичной системе счисления), причем эта 1 не пишется. (23 разряда в float). На знак выделяется 1 бит (0 - положительное).

Порядок (E) - степень базы (двойки) старшего разряда. (8 разрядов в float)

Пример: $237,342=\textit{2,37342}*\textbf{$10^2$}$\\
$(-1)*1.M*2^E$

$-12,345=-1,2345*10^1$ порядок 1\\
$-12,345=-1,2345*10^{-3}$ - порядок 124 (124-127=-3)\\

Сдвиг - возможность записывать отрицательные числа в порядок.

Пример: тип float

Знак $s=0$ - положительное число\\
Порядок $E=01111100_2-127_{10}=-3$\\
Мантисса $M=1.01$ (первая единица неявная)\\
Число $F=1.01*E^{-3}=0,00101=2^{-3}+2^{-5}=0,15625$\\
\\

\textbf{Округление в стандарте IEEE 754}

Чтобы избежать накопления ошибок округления, в ситуациях типа $N.5$ округление идет до ближайшего четного числа. $4.5 \approx 4, 3.5 \approx 4$

В стандарте предусмотрены решения проблем появления большой ошибки, которые возникают при вычитании близких чисел: $x^2-y^2$ автоматически заменяется, например, на $(x-y)(x+y)$.

\textbf{Представимый диапазон вещественных чисел}

$M_0$ - минимальное представимое число. В Matlab $2_{-1074}=4.94...e-324, 2^{-1075}=0$. В Mathcad $M_0=2^{-49}$\\ 
\\
$M_\infty$ - максимальное представимое число. В Matlab $2_{1023}=8.98...E307, 2^{1024}=Inf$. В Mathcad также $M_\infty=2^1023$\\
\\
Машинный эпсилон $\epsilon_m$ - минимальное положительное число, которое при сложении с 1 дает результат, отличный от 1: если $\epsilon<\epsilon_m, 1+\epsilon=1$. В Matlab $1+2^{-52}=1.000..., 1+2^{-53}=1$.

\textbf{Денормализованные числа (subnormal)}

Пусть имеем нормализованное представление с длиной мантиссы |M|=2 бита (плюс 1 бит нормализаци), диапазон значений порядка -1<=E<=2. Можно записать 16 чисел. При этом около 0 будет "дырка" - около 0 значений будет мало.

Что делать? Для обычных больших чисел можно использовать нормализованный формат (мантисса начинается с 1), для маленьких - денормализованный формат (мантисса начинается с 0). При этом Matlab работает с денормализованными числами, а Mathcad их не воспринимает (см. различия в $M_0$).

Ссылка: habrahabr.ru/post/112953/

\section{Влияние порядка выполнения действий}
\begin{enumerate}
\item Умножение\\
Нужно следить, чтобы результаты не выходили за $[M_0, M_\infty], M_0=2^{-u},  M_0=2^{u}$\\
$x_1=2^{u/2}, x_2=2^{u/8}, x_3=2^{3u/4}, x_4=2^{-u/2}, x_5=2^{-3u/4}$\\

$x_1x_2x_3=2^{11u/8}>M_\infty$\\
$x_4x_5=2^{-5u/4}<M_0$\\
$x_1x_2x_3=2^{11u/8}>M_\infty$\\

\item Сложение\\
При сложении больших чисел с малыми точность может теряться: $10^20+1-10^20=0$

Поэтому сначала складываем малые значения, потом прибавляем большие. Если требуется сложить числа в массиве, его нужно сначала отсортировать.
\end{enumerate}

\section{Накопление погрешности округления}

$t$ - число разрядов мантиссы

$a$ - точное число

$\tilde{a}$ - представимое число

Тогда $\frac{|a-\tilde{a}|}{|a|}\leqslant2^{-t}$ - происходит округление.

$fl(a \bigoplus b)=(a \bigoplus b)(1+\epsilon), \epsilon \leqslant 2^{-t}$ - ошибка действия

Пример:

$z=a+b+c$\\

$z_1=fl(a+b)(1+\epsilon_1)$\\

$\tilde{z}=fl(z_1+c)(1+\epsilon_2)=((y_1+y_2)(1+\epsilon_1)+y_3)(1+\epsilon_2)=(y_1+y_2)(1+\epsilon_1)(1+\epsilon_2)+y_3(1+\epsilon_2)$\\

При этом $\tilde{y_1}=y_1(1+\epsilon_1)(1+\epsilon_2), y_2=y_2(1+\epsilon_1)(1+\epsilon_2), y_3=y_3(1+\epsilon_2)$\\

То есть $(y_1+y_2)+y_3 \not = y_1+(y_2+y_3)$

Выводы
\begin{enumerate}
  \item Погрешности вычислений могут накапливаться
  \item Математические законы могут нарушаться
  \item Нельзя рассчитывать на точное равенство
\end{enumerate}

\section{Плохо обусловленные задачи}


to be continued...



\end{document}

