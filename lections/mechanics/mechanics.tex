% !TEX encoding = UTF-8 Unicode
\documentclass[a4paper,12pt]{report}
\newcommand\hmmax{0}
\newcommand\bmmax{0}
		
%Math packages
\usepackage{amsmath, amsfonts, amssymb, amsthm, mathtools}
\usepackage{icomma}
\setcounter{MaxMatrixCols}{20}
%Math packages

%SetFonts
\usepackage{euscript}	 % Шрифт Евклид
\usepackage{mathrsfs} % Красивый матшрифт
\usepackage{dsfont} % Жирный шрифт для множеств чисел
%\usepackage{wasysym} % Для всяких символов
\usepackage{ stmaryrd } %Left \mapsto
\usepackage[T1, T2A]{fontenc}
\usepackage[utf8x]{inputenc}
\usepackage[english, russian]{babel}
%SetFonts

%Graphics
\usepackage{graphicx}
\usepackage{wrapfig}
%Graphics

%Tabular
\usepackage{array, tabularx, tabulary, booktabs}
\usepackage{longtable}
\usepackage{multirow}
%Tabular

%Additional
\newcommand*{\hm}[1]{#1\nobreak\discretionary{}
	{\hbox{$\matsurround=0pt #1$}}{}}
%Additional

%%% Страница
\usepackage{extsizes} % Возможность сделать 14-й шрифт
\usepackage{geometry} % Простой способ задавать поля
\geometry{top=20mm}
\geometry{bottom=20mm}
\geometry{left=15mm}
\geometry{right=15mm}
%
\usepackage{fancyhdr} % Колонтитулы
%\pagestyle{fancy} % Полный набор колонтитулов
\renewcommand{\headrulewidth}{0mm}  % Толщина линейки, отчеркивающей верхний колонтитул
%\lfoot{Нижний левый}
%\rfoot{Нижний правый}
%\rhead{Верхний правый}
%\chead{Верхний в центре}
%\lhead{Верхний левый}
% \cfoot{Нижний в центре} % По умолчанию здесь номер страницы

\usepackage{setspace} % Интерлиньяж
%\onehalfspacing % Интерлиньяж 1.5
%\doublespacing % Интерлиньяж 2
%\singlespacing % Интерлиньяж 1

\usepackage{lastpage} % Узнать, сколько всего страниц в документе.

\usepackage{soul} % Модификаторы начертания

\usepackage{hyperref}
\usepackage[usenames,dvipsnames,svgnames,table,rgb]{xcolor}
\hypersetup{				% Гиперссылки
    unicode=true,           % русские буквы в раздела PDF
    pdftitle={Заголовок},   % Заголовок
    pdfauthor={Автор},      % Автор
    pdfsubject={Тема},      % Тема
    pdfcreator={Создатель}, % Создатель
    pdfproducer={Производитель}, % Производитель
    pdfkeywords={keyword1} {key2} {key3}, % Ключевые слова
    colorlinks=true,       	% false: ссылки в рамках; true: цветные ссылки
    linkcolor=red,          % внутренние ссылки
    citecolor=green,        % на библиографию
    filecolor=magenta,      % на файлы
    urlcolor=cyan           % на URL
}

%\renewcommand{\familydefault}{\sfdefault} % Начертание шрифта



%%%Библиография
\usepackage{cite} % Работа с библиографией
%\usepackage[superscript]{cite} % Ссылки в верхних индексах
%\usepackage[nocompress]{cite} % 
\usepackage{csquotes} % Еще инструменты для ссылок


\usepackage{multicol} % Несколько колонок

\usepackage{tikz} % Работа с графикой
\usepackage{pgfplots}
\usepackage{pgfplotstable}
\usepackage{mathrsfs}
\usetikzlibrary{arrows}

\usepackage{circuitikz}


%reset equation counter after section
\usepackage{chngcntr}

\counterwithin*{equation}{section}
\renewcommand\theequation{\arabic{equation}}
\renewcommand\thesection{\S \arabic{section}}
\renewcommand\thesubsection{\arabic{section}.\arabic{subsection}}
%

\newcommand{\pf}[2]{ 
\dfrac{\partial #1}{\partial #2} 
} 

\newcommand{\df}[2]{ 
\dfrac{d #1}{d #2} 
} 

\newcommand{\sul}[2]{ 
\sum\limits_{#1}^{#2} 
}

%Code listings support
\usepackage{listings}
\lstloadlanguages{[5.2]Mathematica}

\author{Орленко Елена Владимировна}
\title{Классическая механика}
\date{2018 год, весенний семестр}
%\date{}							% Activate to display a given date or no date

\begin{document}

\makeatletter
\@addtoreset{chapter}{part}
\makeatother  

\maketitle

\newpage
%\section{}
%\subsection{}
\setcounter{section}{0}
Область рассмотрения механики - описание систем в отдаленных масштабах с точки зрения микроскопики.

\textbf{Кинематика} - (?) указание траектории всех частиц системы, механическое описание этих систем, а именно: $\vec{r}_i(t), \vec{\upsilon}_i(t)$

Задано: $\ddot{\vec{r}}_i(t)= f(\vec{r}_i(t), \vec{\upsilon}_i(t), t$ - система уравнений движения

Задача кинематики: нахождение неизвестной системы

\textbf{Материальная точка} - абстракция, которой мы можем пренебречь (формой, размерами)

$\vec{\upsilon}_i=\dot{\vec{r}}_i=\frac{d}{dt}\vec{r}_i(t)$

Число независимых степеней свободы не всегда совпадает с числом частиц системы.

Обобщенные координаты системы: $q_1, ..., q_s$ - где $s$ - число степеней свободы.

$\dot{q}_i(t)=\frac{d}{dt}q_i(t)$

$q_1(t), ..., q_s(t), \dot{q}_1(t), ..., \dot{q}_s(t), t$ - полный набор значений для описания состояния системы: все обобщенные координаты и скорости.

\section{Формализм Лагранжа}
\subsection{Принцип Гамильтона (наименьшего действия)}

Функция Лагранжа: $L=(q_1(t), ..., q_s(t), \dot{q}_1(t), ..., \dot{q}_s(t), t)$

Действие функции Лагранжа: $S=\int\limits^{t2}_{t1}L(q_1(t), ..., q_s(t), \dot{q}_1(t), ..., \dot{q}_s(t), t)dt$. Должны быть известны начальные условия $t_1$

Существует единственная истинная траектория, по которой будет происходить движение частицы.

Если $L$ - истинная, то $S$ (действие) имеет экстремальное минимальное значение. $q_i(t)+\delta q_i(t)$

\textbf{Вариация} функции – это небольшое изменение значения функции в точке “x”, не связанное с изменением аргумента.
\begin{equation}
\begin{aligned}
&123\\
&123
\end{aligned}
\end{equation}

\section{Функция Лагранжа для свободного движения}
\subsection{Инерциальные системы}
Требования (для данного вида движения):
\begin{enumerate}
  \item Пространство
  \begin{itemize}
  \item Однородность (свойства одинаковы во всех точках пространства)
  \item Изотропность (одинаково во всех направлениях)
\end{itemize}
  \item Время
  \begin{itemize}
  \item Однородность (тип движения не зависит от точки отсчета времени)
  \item Изотропность (обратимость)
\end{itemize}
\end{enumerate}

При этом $L(\vec{r}, \vec{\upsilon}, t)$:\\
Однородность пространства: $\vec{r}=\vec{r'}+\vec{a} \Rightarrow L(\not \vec{r}, \vec{\upsilon}, t)$ - не зависит от координаты\\
Изотропность пространства: $\Rightarrow L(|\vec{\upsilon}|, t)$\\
Однородность времени: $\Rightarrow L(|\vec{\upsilon}|, \not t)$\\
Изотропность времени: выполняется\\
\\

В итоге $L=L(\upsilon^2)$

$0=\pf{L}{\vec{r}}=\df{}{t} \pf{L}{\vec{\upsilon}} \Rightarrow \pf{L(\upsilon^2)}{\vec{\upsilon^2}} =\vec{C}$ - градиент в поле скоростей\\
\\

$\pf{L(\upsilon^2)}{\vec{\upsilon}}=\pf{L}{\vec{\upsilon}} \pf{\upsilon^2}{\vec{\upsilon}}=\pf{L(\vec{\upsilon})}{\vec{\upsilon}}*2\vec{\upsilon}=\text{const}_t$. Скажем, что $\pf{L(\vec{\upsilon})}{\vec{\upsilon}}=const=\dfrac{m}{2}$

Первый закон Ньютона (Lex Prima): существуют такие системы отсчета, в котором уравнения движения выглядят наиболее просто, то есть, свободно движущаяся частица имеет постоянную скорость: $\vec{\upsilon}=$const

$\vec{\upsilon}=\vec{\upsilon ' }+\vec{u}, \vec{u} $ - скорость движения системы $K'$ относительно $K$. Если $K'$ - инерциальная, то и $K$ - инерциальная. Закон сложения скоростей (Галилея) действует только для нерелятивистских скоростей. Он также предполагает абсолютность времени (одинаковое течение во всех СО), что неверно в рамках теории относительности.

\subsection{Функция Лагранжа для свободно движущейся частицы}
Пусть одна СО движется относительно другой с малой скоростью $\vec{\delta}$: $\vec{\upsilon'}=\vec{\upsilon}+\vec{\delta}$. Тогда:

$L({\upsilon'}^2)=L(\upsilon^2+2\vec{\upsilon}\vec{\delta}+\delta^2)=L({\upsilon'}^2)+2\vec{\upsilon'}\vec{\delta} \frac {\partial{L}}{\partial{\upsilon^2}}=L({\upsilon'}^2)+\frac{d}{dt}(2\vec{r'}\vec{\delta} \frac {\partial{L}}{\partial{\upsilon^2}}$.\\

Если $\frac {\partial{L}}{\partial{\upsilon^2}}=const=\frac{m}{2}$, эти две функции Лагранжа $L(\upsilon^2)$ и $L({\upsilon'}^2)$ эквивалентны. $L(\upsilon^2)=\frac{m\upsilon^2}{2}$.\\

Требование малости $\vec{\delta}$ необязательно: пусть $\vec{{\upsilon'}}=\vec{\upsilon}+\vec{V}$, тогда $\frac{m\upsilon^2}{2}=\frac{m{\upsilon'}^2}{2}+\frac{m\vec{\upsilon}\vec{V}}{2}+\frac{mV^2}{2}$\\

\textbf{Обобщенные координаты}

Система свободных невзаимодействующих частиц:

$L=({\upsilon_1}^2, ..., {\upsilon_n}^2)=\sum\frac{m{\upsilon_i}^2}{2}$

Пусть есть обобщенные координаты: $q_1...q_k$: можем через них выразить старые координаты $x_i=f_i(q_1, ...,  q_n) \to \dot{x}=\sum\frac{\partial f_{ik}}{\partial q_k}\dot{q_k}$

Тогда $L=\sum\frac{m_i}{2} \sum\frac{\partial f_{ik}}{\partial q_k}\dot{q_k} \sum\frac{\partial f_{il}}{\partial q_l}\dot{q_l}=\sum\frac{1}{2}\dot{q_k}\dot{q_l}\sum \frac{m_i}{2} \frac{\partial f_{il}}{\partial q_k}\frac{\partial f_{ik}}{\partial q_k}=\sum\frac{a_{kl}}{2}\dot{q}_k\dot{q}_l$\\

Замечание: что такое функция действия? $L=\frac{m\upsilon^2}{2}$\\

$S=\int^{t2}_{t1} \frac{m\upsilon^2}{2}dt=\frac{m}{2}\int^{t2}_{t1}\left( \frac{dl}{st}\right)^2=\upsilon\frac{m}{2}\int^{t2}_{t1}\frac{dl}{dt}dl$ - проще всего двигаться прямолинейно, а не по дуге\\

Длина дуги и скорость в \begin{enumerate}
  \item декартовой системе: $(dl)^2=(dx)^2+(dy)^2+(dz)^2$
  
  $\upsilon^2=\upsilon_x^2+\upsilon_y^2+\upsilon_z^2$
  
  \item цилиндрической системе:  $(dl)^2=(dz)^2+(d\rho)^2+(\rho\dot{\varphi})^2$
  
  $\upsilon^2=\upsilon_z^2+(\dot{\rho})^2+(\rho\dot{\varphi})^2$
  
  \item сферической системе: ...
  
  \section{Функция Лагранжа системы взаимодействующих частиц}
  \textbf{Движение под действием консервативной силы $\vec{F}$}
Для свободного движения:

  $\frac{d}{dt}\frac{\partial{L}}{\partial{\upsilon^2}}=\frac{\partial{L}}{\partial{\vec{r}}}$

  Могут быть другие компоненты:
  
  $L=T+...; T=\frac{m\upsilon^2}{2}$  
  
Тогда     $\frac{d}{dt}\frac{\partial{T}}{\partial{\upsilon^2}}=\frac{\partial{T}}{\partial{\vec{r}}} \to \frac{d}{dt}(m\vec{\upsilon})=\frac{\partial{f(\vec{r})}}{\partial{\vec{r}}}$

При этом $\frac{d}{dt}(m\vec{\upsilon})=0$   
\end{enumerate}


























\end{document}

