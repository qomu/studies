% !TEX encoding = UTF-8 Unicode
\documentclass[a4paper,12pt]{report}
\newcommand\hmmax{0}
\newcommand\bmmax{0}
		
%Math packages
\usepackage{amsmath, amsfonts, amssymb, amsthm, mathtools}
\usepackage{icomma}
\setcounter{MaxMatrixCols}{20}
%Math packages

%SetFonts
\usepackage{euscript}	 % Шрифт Евклид
\usepackage{mathrsfs} % Красивый матшрифт
\usepackage{dsfont} % Жирный шрифт для множеств чисел
%\usepackage{wasysym} % Для всяких символов
\usepackage{ stmaryrd } %Left \mapsto
\usepackage[T1, T2A]{fontenc}
\usepackage[utf8x]{inputenc}
\usepackage[english, russian]{babel}
%SetFonts

%Graphics
\usepackage{graphicx}
\usepackage{wrapfig}
%Graphics

%Tabular
\usepackage{array, tabularx, tabulary, booktabs}
\usepackage{longtable}
\usepackage{multirow}
%Tabular

%Additional
\newcommand*{\hm}[1]{#1\nobreak\discretionary{}
	{\hbox{$\matsurround=0pt #1$}}{}}
%Additional

%%% Страница
\usepackage{extsizes} % Возможность сделать 14-й шрифт
\usepackage{geometry} % Простой способ задавать поля
\geometry{top=20mm}
\geometry{bottom=20mm}
\geometry{left=15mm}
\geometry{right=15mm}
%
\usepackage{fancyhdr} % Колонтитулы
%\pagestyle{fancy} % Полный набор колонтитулов
\renewcommand{\headrulewidth}{0mm}  % Толщина линейки, отчеркивающей верхний колонтитул
%\lfoot{Нижний левый}
%\rfoot{Нижний правый}
%\rhead{Верхний правый}
%\chead{Верхний в центре}
%\lhead{Верхний левый}
% \cfoot{Нижний в центре} % По умолчанию здесь номер страницы

\usepackage{setspace} % Интерлиньяж
%\onehalfspacing % Интерлиньяж 1.5
%\doublespacing % Интерлиньяж 2
%\singlespacing % Интерлиньяж 1

\usepackage{lastpage} % Узнать, сколько всего страниц в документе.

\usepackage{soul} % Модификаторы начертания

\usepackage{hyperref}
\usepackage[usenames,dvipsnames,svgnames,table,rgb]{xcolor}
\hypersetup{				% Гиперссылки
    unicode=true,           % русские буквы в раздела PDF
    pdftitle={Заголовок},   % Заголовок
    pdfauthor={Автор},      % Автор
    pdfsubject={Тема},      % Тема
    pdfcreator={Создатель}, % Создатель
    pdfproducer={Производитель}, % Производитель
    pdfkeywords={keyword1} {key2} {key3}, % Ключевые слова
    colorlinks=true,       	% false: ссылки в рамках; true: цветные ссылки
    linkcolor=red,          % внутренние ссылки
    citecolor=green,        % на библиографию
    filecolor=magenta,      % на файлы
    urlcolor=cyan           % на URL
}

%\renewcommand{\familydefault}{\sfdefault} % Начертание шрифта



%%%Библиография
\usepackage{cite} % Работа с библиографией
%\usepackage[superscript]{cite} % Ссылки в верхних индексах
%\usepackage[nocompress]{cite} % 
\usepackage{csquotes} % Еще инструменты для ссылок


\usepackage{multicol} % Несколько колонок

\usepackage{tikz} % Работа с графикой
\usepackage{pgfplots}
\usepackage{pgfplotstable}
\usepackage{mathrsfs}
\usetikzlibrary{arrows}


%reset equation counter after section
\usepackage{chngcntr}

\counterwithin*{equation}{section}
\renewcommand\theequation{\arabic{equation}}
%

\renewcommand\thesection{\arabic{section}}
%Code listings support
\usepackage{listings}
\lstloadlanguages{[5.2]Mathematica}
\usepackage{circuitikzgit}



\author{Сочава Александр Андреевич}
\title{Радиотехнические цепи и сигналы}
\date{2018 год, весенний семестр}
%\date{}							% Activate to display a given date or no date

\begin{document}

\makeatletter
\@addtoreset{chapter}{part}
\makeatother  

\maketitle
\newpage
%\section{}
%\subsection{}

\begin{center}
\textbf{ПРЕДИСЛОВИЕ}
\end{center}

	В данном сборнике представлены описания и методические указания к трем лабораторным работам. Материал пособия содержит сведения теоретического и практического характера, достаточные для выполнения этих работ.
	
Лабораторные работы посвящены колебательным процессам в простейших электрических цепях: $RC$-фильтрах и $LC$-контурах. Их выполнение позволяет студентам глубже понять физические основы изучаемых явлений и закономерности, которым они подчиняются.

	В первой работе исследуются $RC$-цепи, которые весьма часто используются в качестве фильтров нижних и верхних частот. Эта работа построена так, что она может выполняться без предварительной подготовки на первом (вводном) занятии. Одновременно студенты знакомятся с измерительными приборами, которые широко применяются в последующих работах.
	
	Во второй работе изучаются вынужденные колебания в колебательном $LC$-контуре: резонансные характеристики последовательного и параллельного контуров, их зависимость от физических параметров элементов схемы, возможность использования $LC$-контура в качестве частотно-избирательной цепи.
	
	В третьей работе исследуются свободные колебания в одиночном $LC$-контуре и в системе двух индуктивно связанных контуров. Изучаемые здесь явления имеют принципиальное значение не только в радиосхемах, но и в устройствах СВЧ, оптического диапазона, а также в явлениях, происходящих на молекулярном уровне.
	
	Настоящее пособие является переизданием ранее выпущенного пособия «Электрические колебания в линейных цепях»; авторы А. Д. Жуков, Э. Ф. Зайцев, Б. А. Мартынов, Ю. Н. Новиков, Л.:ЛГТУ, 1991. В данное издание внесены лишь небольшие изменения.
\newpage
\begin{center}
Лабораторная работа\\
\textbf{ИССЛЕДОВАНИЕ ПАССИВНЫХ ЦЕПЕЙ}
\end{center}

	В данной работе изучаются $RC$-цепи, часто применяемые в электронных устройствах. в качестве простейших разновидностей электрических фильтров нижних и верхних частот. Объектом исследования является, кроме того, параллельный колебательный контур.
	
	Цель лабораторной работы -- умение проводить изменения при помощи приборов, используемых в учебной лаборатории, а также применять различные методы сняти частотных и временн$\acute{\text{ы}}$х характеристик простейших пассивных цепей и определения параметров цепей и сигналов.
\begin{table}[h!]
  \begin{center}
    \caption{Назначение и погрешности измерительных приборов.}
    \label{tab:table1}
    \begin{tabular}{| m{12em} | m{12em} | m{14em} |}
    \hline
     Название & Назначение & Погрешность \\
      \hline
      Милливольтметр В3-38 & Измерение действующего значения синусоидального напряжения & Основная погрешность прибора не выше 6\% от конечного значения установленного предела измерения\\
      \hline
      Генераторы сигналов звуковых и ультразвуковых частот Г3-33, Г3-112 & Источник гармонических колебаний & Основная приведенная погрешность по частоте не превышает $(0,02f+1\text{ Гц})$\\
      \hline
      Частотомер Ч3-33 & Измерение частоты электрических колебаний & Основная погрешность измерения частоты не превышает $1/t$, где t - время измерения (счета)\\
      \hline
      Осциллографы универсальные С1-68, С1-83 & Исследование формы электрических колебаний, измерение напряжений сигналов, измерение временных интервалов & Основная погрешность измерения напряжений при размере изображения от 2 до 6 делений шкалы экрана не превышает $8\%$. Основная погрешность измерения временн$\acute{\text{ы}}$х интервалов при размере изображения по горизонтали от 4 до 8 делений не превышает $8\%$\\
      \hline
    \end{tabular}
  \end{center}
\end{table}

	Варианты исследуемых цепей собираются на монтажной плате посредством установки сменных вставок, в которые вмонтированы отдельные элементы: катушка индуктивности, резисторы, конденсаторы.
	
	На вставках указаны номинальные значения параметров элементов (в том числе сопротивление потерь катушки). Допустимые отклонения от номиналов не превышают $10\%$.
	
\begin{center}
\textbf{Порядок выполнения работы}\\
\textit{Однозвенный $RC$-фильтр в режиме гармонических колебаний}
\end{center}

1. Используя схему измерений, приведенную на рис. 1, снимите амплитудно-частотные характеристики двух вариантов фильтров нижних частот (ФНЧ):

а) $R = 51$ кОм, $C = 1300$ пФ; б) $R = 51$ кОм, $C = 10$ нФ;

\begin{center}
\begin{circuitikz}[american voltages, american currents, european resistors]
\draw (-2,0) to[twoport, text=Генератор] (-2,4)--(1, 4) to[twoport, text=Частотомер] (1, 0)--(-2, 0);
\draw (1, 4)--(3, 4) to[voltmeter, v^<=$u_\text{вых}$] (3, 0)--(1, 0);
\draw (3, 4) to[R, l=$R$] (6, 4) to[C, l=$C$] (6,0)--(3, 0);
\draw (6, 4)--(8, 4) to[voltmeter, v^<=$u_\text{вых}$] (8, 0)--(6, 0);
\draw (8, 4)--(11, 4) to[twoport, text=Осциллограф] (11, 0)--(8, 0);
\end{circuitikz}
\end{center}

	Амплитудно-частотная характеристика (АЧХ) - зависимость модуля комплексного коэффициента передачи фильтра $K=U_{вх}/U_{вых}$ от частоты $f$. (Здесь $U_\text{вх}$ и $U_\text{вых}$ - действующие значения входного и выходного напряжений). При снятии АЧХ целесообразно поддерживать $U_\text{вх}$ постоянным и равным, например, $0,1$ В или $1$ В; (такие значения удобны при вычислении K). Значения $f$ выбирайте таким образом, чтобы при построении АЧХ в полулогарифмическом масштабе обеспечивалось достаточно равномерное расположение отсчетов по частотной оси (пример на рис. 2).
	
2. Для обоих вариантов ФНЧ проведите измерения, необходимые для нахождения граничных частот $f_c$, на которых $K$ равен $1/\sqrt{2}=0,707$ от наибольшего значения.

3. При $R = 51$ кОм, $C = 10$ нФ снимите фазочастотную характеристику ФНЧ.

Фазочастотная характеристика (ФНЧ) - зависимость сдвига фаз $\varphi$ между выходным и входным напряжениями от частоты.

Для нахождения сдвига фаз $\varphi$ переведите осциллограф в режим «$X-Y$» (при этом выключается периодическая развертка луча по горизонтали); подайте напряжение с выхода ФНЧ на вход «$Y$» осциллографа, а входное - на вход «$X$»; изменяя амплитуду входного напряжения и коэффициент отклонения канала «$Y$», добейтесь, чтобы получающаяся на экране фигура Лиссажу (рис. 3) охватывала б$\acute{\text{о}}$льшую часть экрана; совместите центр эллипса с центром масштабной сетки на экране. Сдвиг фаз $\varphi$ находится при помощи соотношения
\begin{equation}
	\begin{aligned}
		&|\!\sin{\varphi}| = X_0/A=Y_0/B, \notag
	\end{aligned}
\end{equation}
где $X_0$ и $A$ или $Y_0$ и $B$ (рис. 3) измеряются в делениях масштабной сетки осциллографа.

Подберите частоту колебаний, при которой $\varphi=45^{\circ}$, и сравните ее со значеним граничной частоты, найденном в п. 2. Постройте ФЧХ в полулогарифмическом масштабе на одном графике с АЧХ.

4. Рассчитайте граничные частоты ФНЧ по формуле $f_c=1/(2\pi RC)$. Результаты вычислений сравните с измеренными значениями (см. п. 2), сведя расчетные и экспериментальные значения в таблицу.
\begin{center}
\textit{Однозвенный $RC$-фильтр в режиме гармонических колебаний}
\end{center}

 5. Переведя осциллограф в режим периодической развертки, подайте на вход импульсное напряжение с гнезд «$\textrm{П}$» и «$\bot$», расположенных в нише на боковой стенке прибора. (Источник гармонического напряжения откючите).

Используя калибровочные значения коэффициентов отклонения луча по горизонтали (например, $0,1$ мс/см) и по вертикали (например, $0,5$ В/см), измерьте период повторения импульсов $T$ и их амплитуду $E$ на входе (рис. 4-а). Перед измерениями необходимо установить ручки плавной регуляции коэффициентов отклонения луча и скорости развертки в крайне правое положение (крайне правое при вращении по часовой стрелке). Вычислите частоту следования импульсов $F=1/t$ и сравните ее с измеренной частотомером.
6. Измерьте значения амплитуды (размаха) «$A$» импульсов на выходе ФНЧ (рис. 4-б) при двух наборах параметров:

а) $R = 51$ кОм, $C = 1300$ пФ; б) $R = 51$ кОм, $C = 10$ нФ.

7. При $R = 51$ кОм, $C = 1300$ пФ измерьте длительность фронта выходного импульса $\tau_\text{ф}=2,2RC$.

8. При тех же параметрах, что и в п. 6, зарисуйте осциллограммы входного и выходного напряжений. Отметьте на осциллограммах измереннное значение амплитуд, периода и длительности фронта (см. п. 7).

Убедитесь, что при $R = 51$ кОм, $C = 10$ нФ ФНЧ ведет себя как интегратор.

9. Используя полученные в п. 5 результаты измерений $T$ и $E$, при помощи формулы $A=E\text{th}(T/(4RC))$ (см. приложение) по заданным $R$ и $C$ вычислите значения амплитуды (размаха) $A$ и сравните их с найденными в п. 6, сведя в таблицу результаты расчета и эксперимента.
 
10. Поменяв местами резистор и конденсатор (при $C=1300$ пФ), получите и зарисуйте осциллограмму напряжения на выходе фильтра верхних частот (ФВЧ). Для импульса на выходе ФВЧ проведите измерение длительности $\tau_\text{и}$ по уровню $0,1$ амплитуды (см. рис. 5) и сравните полученное значение с рассчитанным по формуле $\tau_\text{и}=2,3 RC$.

11. Для $R = 51$ кОм, $C = 1300$ пФ, используя экспериментально полученные для ФНЧ значения граничной частоты $f_c$ и длительности фронта $\tau_\text{ф}$, найдите произведение $f_c \tau_\text{ф}$ и сравните его с теоретическим значением $0,35$.

\begin{center}
\textit{Параллельный колебательный контур}
\end{center}
12. Соберите на монтажной плате параллельны колебательный контур, как показано на рис. 6. Перестраивая по частоте источник гармонических колебаний, найдите частоту $f_0$, при которой наблюдается резонанс в параллельном контуре.

\begin{center}
\begin{circuitikz}[american voltages, american currents, european resistors]
\draw (-1,0) to[twoport, text=ГСС] (-1,4)--(1, 4) to[twoport, text=Частотомер] (1, 0)--(-1, 0);
\draw (1, 4)--(3, 4) to[voltmeter, v^<=$u_\text{вых}$] (3, 0)--(1, 0);
\draw (3, 4) to[R, l=$470\text{ кОм}$] (6, 4) to[L, l=$L$] (6, 0)--(3, 0);
\draw (6, 4)--(8, 4) to[C=1300pF, l=$\!1300\text{ пФ}$] (8,0)--(6, 0);
\draw (8, 4)--(10.5, 4) to[voltmeter, v^<=$u_\text{вых}$] (10.5, 0)--(8, 0);
\end{circuitikz}
\end{center}

Установите максимальное напряжение на контуре при резонансе равным $10$ мВ. Измерьте напряжение на выходе генератора. Поддерживая его неизменным, найдите при помощи частотомера, подключенного к выходу генератора, верхнюю ($f_\text{в}$) и нижнюю ($f_\text{н}$) границы полосы пропускания контура (по уровню $1/\sqrt{2}=0,707$ от резонансного значения напряжения).

По формуле $Q=f_0/(f_\text{в}-f_\text{н})$ вычислите добротность контура.

В диапазоне частот, при которых напряжение на контуре не ниже $30\%$ от максимума, снимите резонансную кривую (рис. 7). 

13. Считая известными параметры катушки индуктивности (рис. 6-б), рассчитайте резонансную частоту и добротность контура по формулам:
\begin{equation}
	\begin{aligned}
		&f_0=\frac{1}{2\pi \sqrt{L(C+C_0+C_\text{п})}}, \\
		&Q=\frac{2\pi f_0 L}{r}, \notag
	\end{aligned}
\end{equation}
где $L$ - индуктивность катушки, r -  сопротивление потерь (параметры $L$ и $r$ указаны на держателе катушки); $C_0$ - собственная емкость катушки ($C_0 \approx 40$ пФ); $C_\text{п}$ - входная емкость вольтметра В3-38 с учетом кабеля ($C_\text{п} \approx 60$ пФ).

Сравните расчетные и экспериментальные значения $f_0$ и $Q$, сведя их в таблицу.


\begin{center}
\textbf{Приложение}\\
\textit{Вывод формулы для амплитуды импульсов на емкости}
\end{center}

На основании второго закона Кирхгофа для цепи, представленной на рис. 8, справедливо равенство
\begin{equation}
	\begin{aligned}
		&u_C + u_R = e. \notag
	\end{aligned}
\end{equation}

\begin{center}
\begin{circuitikz}[american voltages, american currents, european resistors]
\draw (0,0) to[I, v=$e$] (0,2);
\draw (0,2) to[R] (2,2);
\draw (2,2) to[C] (2,0)--(0,0);
\end{circuitikz}
\end{center}


Отсюда, а также из соотношений, связывающих токи и напряжения на элементах $R$ и $C$, можно получить дифференциальное уравнение
\begin{equation}
	\begin{aligned}
		&\frac{du_R}{dt}+\frac{u_R}{RC}=\frac{de}{dt}, \notag
	\end{aligned}
\end{equation}
которым описываются процессы в анализируемой цепи. 

Для моментов времени $t_0$ и $t$, принадлежащей какому-либо интервалу, в течение которого ЭДС $e$ остается неизменной ($de/dt=0$), из полученного уравнения следует, что
\begin{equation}
	\begin{aligned}
		&u_R(t)=u_R(t_0)\text{exp}\left(\frac{t_0-t}{RC}\right).
	\end{aligned}
\end{equation}

Предположим, что входное напряжение (рис. 9-а) не содержит постоянной составляющей и при $nT<t<(n+0,5)T$ принимает значение $E/2$, а при $(n-0,5)T<t<nT$ $-E/2$. (Здесь $n$ - любое целое число.) В установившемся режиме постоянные составляющие напряжений на сопротивлении и емкости также равняются нулю, а значения каждого из этих напряжений в моменты времени, отстоящие друг от друга на нечетное число полупериодов, различаются только знаком (рис. 9-б, в).

Моменты времени $t=nT/2$ необходимо рассматривать особо. В эти моменты напряжение $u_R$ получает скачкообразное приращение на $+E$ или $-E$ в зависимости от того, увеличивается или уменьшается входное напряжение.

С учетом сказанного, обозначая через $u_R(0-)$ и $u_R(0+)$ значения напряжения на сопротивлении при $t=0$ до и после скачка и используя (2), получаем
\begin{equation}
	\begin{aligned}
		&u_R(0-)=u_R(0+)-E=-u_R(0+)\text{exp}\left(\frac{-T}{2RC}\right),\notag
	\end{aligned}
\end{equation}
откуда
\begin{equation}
	\begin{aligned}
		&u_R(0+)=\frac{E}{1+\text{exp}\left(\frac{-T}{2RC}\right)}.
	\end{aligned}
\end{equation}

Выражая далее напряжение на емкости при $t=nT/2$ через $u_R$ и $e$ согласно (1), получим
\begin{equation}
	\begin{aligned}
		&\frac{A}{2}=\frac{E}{2}-u_R(0+)\text{exp}\left(\frac{-T}{2RC}\right).\notag
	\end{aligned}
\end{equation}

и принимая во внимание (3), придем к следующей формуле для амплитуды (размаха) импульсов на емкости
\begin{equation}
	\begin{aligned}
		&A=E \frac{1-\text{exp}\!\left(\frac{-T}{2RC}\right)}{1+\text{exp}\left(\frac{-T}{2RC}\right)}=E\text{th}\!\left(\frac{T}{4RC}\right).\notag
	\end{aligned}
\end{equation}

Полученное выражение справедливо и в случае ненулевых постоянныз составлющиз напряжений (см., например, рис. 4).

С помощью приведенных выше соотношений можно получить также формулы для $\tau_\text{ф}$ в случае ФНЧ и $\tau_\text{и}$ в случае ФВЧ.







\newpage
\begin{center}
\begin{circuitikz} \draw
(0,0) node[draw,minimum width=2cm,minimum height=2.4cm] (c) {}
($(c.west)!0.5!(c.north west)$) coordinate (cnw)
($(c.west)!0.5!(c.south west)$) coordinate (csw)
($(c.east)!0.5!(c.north east)$) coordinate (cne)
($(c.east)!0.5!(c.south east)$) coordinate (cse)
(cnw) to[short,-o] ++(-1,0)
(cne) to[short,-o] ++(1,0)
(cse) to[short,-o] ++(1,0)
(csw) to[short,-o] ++(-1,0)
(cnw) node[above left] {$1$}
(csw) node[above left] {$1'$}
(cne) node[above right] {$2$}
(cse) node[above right] {$2'$}
;
\end{circuitikz}
\end{center}




















\end{document}